documentclass[11pt]{article}

\begin{document}

\title{The cost of cloud computing for {\em de novo} transcriptome 
assembly on cloud computers}
\author{Leigh Sheneman and C. Titus Brown}
\maketitle

\section*{Abstract}

\section*{Introduction}

Within biology, the number and size of sequencing data sets is
increasing rapidly, leading to an increased demand for computational
infrastructure. Many researchers and institutions are finding it
difficult to cope with this increased demand.  One possible solution
is to rent compute power on demand from a commercial provider -- so
called ``cloud computing.'' Cloud computing has been proposed as an
alternative to purchasing and maintaining dedicated servers with
technology that quickly becomes outdated.

Cloud computing offers many advantages both for researchers and for
teaching workshops; cloud providers such as Amazon Web Services,
Microsoft Azure, Google Compute Engine, and Rackspace Cloud can scale
with hourly demand and provide a range of hardware and software
configurations.  Moreover, cloud computing offers an opportunity for
reproducibility in computational biology, because the exact execution
environment behind a paper can be made available to readers.

However, cloud computing resources can also be a poor fit for
computational research.  Cloud computing relies on virtualization
technology to provide a configurable execution environment to users;
this yields poorer computational performance than execution on non-
virtualized hardware with the same specifications.  Moreover, cloud
computing providers do not typically have specialized hardware or
computers with substantial amounts of random-access memory (RAM); for
example, Google Compute Engine currently limits computers to 104
gigabytes (GB) of RAM. And finally, cloud computing costs can be
unpredictable: because charges are based on usage, processing tasks
that take longer cost more, and the length of a task cannot always be
predicted in advance.

Below we evaluate the computational requirements and overall cost of a
data- and memory-intensive work-flow for {\em de novo} reconstruction
of transcriptomes from Illumina sequencing reads, running on a
published {\em Nematostella} data set.  We show that different steps
in the work-flow have dramatically different computational
requirements from one another, and that substantial differences in the
time to run the work-flow can be achieved by choosing different
providers or hardware. We also compare and contrast the cost of
sequencing with the cost of the associated computation.

\section*{Results}

Our sequence analysis work-flow has multiple steps, including primer
trimming, quality filtering, data preprocessing with digital
normalization, primary data processing in the form of assembly,
annotation against existing databases, and correlation of the assembly
results with the original data via mapping.  Each of these steps has
its own computational requirements: primer trimming and quality
filtering are primarily CPU- and disk-intensive, while digital
normalization and assembly are memory intensive, annotation is CPU-
intensive, and mapping is both CPU- and disk- intensive.  These are
unavoidable given the underlying algorithms.

(@LS, how can we best measure io wait? @CTB vmstat -- time spent
waiting between recursive calls;  sar --  \%wio from sar -u -> i/o
bound time;   avwait to avserv form sar -d will show bottleneck;
(user + nice + system) cpu usage)

Transcriptome assembly relies on several memory intensive steps, with
only sporadic intensive disk use and relatively low levels of CPU
usage compared to a mapping work-flow. The memory requirements for the
Trinity assembler are hard to predict a priori because they depend on
biological features of the data -- here, the underlying information
content of the transcriptome being sequenced, together with the errors
remaining after digital normalization.  This is a common feature of De
Bruijn graph assemblers, whose memory use scales with the sum of
information content and erroneous sequence.  In practice this means
that large transcriptomes or transcriptomes from multiple species may
require substantially more memory to assemble than small, single-
species transcriptomes.

The cost of computation for these data sets is approximately \$100 for
each of the configurations (see table XX), with lower costs typically
incurring higher overall compute time and delaying delivery of
results. Regardless, the cost of computation for the data sets is
dwarfed by the expense of generating the data: \$1000 of data needs
about \$100 of computing.

Since the configurations are rented as a package, for example, one
cannot simply switch to a larger hard drive, you must reconfigure your
machine to the appropriate tier. AWS has 22 of these ``tiers'' in the
current generation of hardware alone. Their prices range from \$0.02
to \$6.82 per hour, thus proper resource estimation is vital for users
with limited fund availability.

A minimal requirement is to be able to provide enough disk space for a
data set to run through the protocol with out depletion. Even the
Nematostella data we used needs 200GB to complete the pipeline, thus
ruling out exclusively using an AWS m1.xlarge instance.


@ctb the next paragraph hints at what you discuss belong. Trying to
@gage if it is more suiting here or in the discussion. In fact I have
@a feeling plugging in our data may change the structure.

AWS, Rackspace and GCE all use similar hardware, the physical
implementation of the cluster is a factor in performance. When renting
from Rackspace the resources used by your virtual machine are in the
same rack. On the other hand AWS’s instances are connected over the
network. Intuitively, when all the components are in close proximity
you experience higher input rates than if the data has to travel
through 20 feet of networking cable. It also stands to reason, if our
algorithms and data structures are encapsulated  in the the CPU cache
the performance will improve an order of magnitude.

\section*{Discussion}

It is difficult predict the performance of our full de novo
transcriptome assembly work-flow on a new computer hardware
configuration, given the many different stages, each with their
own distinct computational requirements.

Optimizing work-flows for multiple cloud providers is therefore
inherently challenging.  In addition to different base configurations
of disk and CPU performance and memory size, cloud providers may be
running virtual machines on heterogeneous hardware, and sharing disk
and network bandwidth with other applications.  Each cloud provider
will also have a distinct set of costs associated with each of their
hardware configurations (e.g. Amazon, network, EBS), with non-obvious
implications for the work-flow.  And each cloud provider typically
makes available multiple configurations of CPU, disk, and memory, each
of which will have its own performance characteristics.  This makes
choosing the optimum computing platform a difficult multidimensional
optimization problem.

By establishing the limiting factor for each step of our assembly
protocol  we have discovered bounds for platform selection for that
step. With the help of virtualization, it is easy to switch between
configurations at each juncture. Thus, we can reduce the complexity
of the optimization problem by focusing on the particular computation
needs at each step. In addition, if it is found the data reaches a
bottleneck a few steps in, we can reconfigure the setup and proceed
with a minimal loss of time. This process can result in a  substantial
reduction of cost, since we can make an educated guess what the  most
fitting CPU, disk, and memory requirements will be then adjust to a
larger machine if need be.

Discuss ``surfacing'' otherwise hidden costs from HPCs.

(@CTB I assume you are referring to maintenance, energy cost, etc?)

\end{document}
